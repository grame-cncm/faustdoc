\chapter{Introduction}
\label{introduction}


\faust (\textit{Functional Audio Stream}) is a functional programming language specifically designed for real-time signal processing and synthesis.  \faust targets high-performance signal processing applications and audio plug-ins for a variety of platforms and standards. 

\section{Design Principles} 

Various principles have guided the design of \faust :

\begin{itemize}

\item \faust is a \textit{specification language}. It aims at providing an adequate notation to describe \textit{signal processors} from a mathematical point of view. \faust is, as much as possible, free from implementation details. 

\item \faust programs are fully compiled, not interpreted. The compiler translates \faust programs into equivalent target languages such as C/C++, LLVM IR, WebAssembly, and others, taking care of generating the most efficient code. The result can generally compete with, and sometimes even outperform, C++ code written by seasoned programmers.

\item The generated code works at the sample level. It is therefore suited to implement low-level DSP functions like recursive filters. Moreover the code can be easily embedded. It is self-contained and does not depend on any DSP library or runtime system. It has a very deterministic behavior and a constant memory footprint. 

\item The semantics of \faust are simple and well defined. This is not just of academic interest. It allows the \faust compiler to be \emph{semantically driven}. Instead of compiling a program literally, it compiles the mathematical function it denotes. This feature is useful, for example, to promote component reuse while preserving optimal performance.  

\item \faust is a textual language but nevertheless block-diagram oriented. It actually combines two approaches: \textit{functional programming} and \textit{algebraic block-diagrams}. The key idea is to view block-diagram construction as function composition. For that purpose, \faust relies on a \emph{block-diagram algebra} of five composition operations (\lstinline': , ~ <: :>').

\item Thanks to the notion of \textit{architecture}, \faust programs can be easily deployed on a large variety of audio platforms and plugin formats without any change to the \faust code.

\end{itemize}

\section{Signal Processor Semantics}
A \faust program describes a \emph{signal processor}. 
The role of a \textit{signal processor} is to transform a (possibly empty) group of \emph{input signals} in order to produce a (possibly empty) group of \emph{output signals}. 
Most audio equipment can be modeled as \emph{signal processors}. 
They have audio inputs, audio outputs, as well as control signals interfaced with sliders, knobs, vu-meters, etc. 

More precisely:

\begin{itemize}

\item A \emph{signal} $s$ is a discrete function of time $s:\mathbb{Z}\rightarrow\mathbb{R}$.
\marginpar{\faust considers two types of signals: \emph{integer signals} ($s:\mathbb{Z}\rightarrow\mathbb{Z}$) and \emph{floating point signals} ($s:\mathbb{Z}\rightarrow\mathbb{Q}$). Exchanges with the outside world are, by convention, made using floating point signals. The full range is represented by sample values between -1.0 and +1.0.}
The value of a signal $s$ at time $t$ is written $s(t)$. The values of signals are usually needed starting from time $0$. But to take into account \emph{delay operations}, negative times are possible and are always mapped to zeros. Therefore for any \faust signal $s$ we have $\forall t<0, s(t)=0$. In operational terms this corresponds to assuming that all delay lines are signals initialized with $0$s.
 
\item The set of all possible signals is $\mathbb{S}=\mathbb{Z}\rightarrow\mathbb{R}$.

\item A group of $n$ signals (a \emph{n}-tuple of signals) is written
$(s_{1},\ldots,s_{n})\in \mathbb{S}^{n}$.
The \emph{empty tuple}, single element of $\mathbb{S}^{0}$, is denoted $()$.

\item A \emph{signal processor} $p$ is a function from
\emph{n}-tuples of signals to \emph{m}-tuples of signals
$p:\mathbb{S}^{n}\rightarrow\mathbb{S}^{m}$. The set $\mathbb{P}=\bigcup_{n,m}\mathbb{S}^{n}\rightarrow\mathbb{S}^{m}$ is the
set of all possible signal processors.

\end{itemize}

As an example, let's express the semantics of the \faust primitive \lstinline'+'. Like any \faust expression, it is a signal processor. Its signature is $\mathbb{S}^{2}\rightarrow\mathbb{S}$. It takes two input signals $X_0$ and $X_1$ and produces an output signal $Y$ such that $Y(t) = X_0(t)+X_1(t)$. 

Numbers are signal processors too. For example the number $3$ has signature $\mathbb{S}^{0}\rightarrow\mathbb{S}$. It takes no input signals and produces an output signal $Y$ such that $Y(t) = 3$. 
