%--------------------------------------------------------------------------------------------------------------
\section{Primitives}
%--------------------------------------------------------------------------------------------------------------
\label{primitives}
The primitive signal processing operations represent the built-in functionalities of \faust, that is the atomic operations on signals provided by the language. All these primitives denote \emph{signal processors}, functions transforming \emph{input signals} into \emph{output signals}.

\begin{rail}
	primitive : number
	| route
	| waveform
	| soundfile
	| cprimitive
	| mathprimitive
	| delayandtables
	| uielements
	| modulation
	;
\end{rail}

%--------------------------------------------------------------------------------------------------------------
\subsection{Numbers}
%--------------------------------------------------------------------------------------------------------------

\faust considers two types of numbers: \textit{integers} and \textit{floats}. Integers are implemented as signed 32-bit integers, and floats are implemented with single, double, or extended precision depending on the compiler options. Floats are available in decimal or scientific notation.

\begin{rail}
	int : (|'+'|'-')(digit+) ;
\end{rail}

\begin{rail}
	float : (|'+'|'-')( ((digit+)'.'(digit*)) | ((digit*) '.' (digit+)) )(|exponent);
\end{rail}

\begin{rail}
	exponent : 'e'(|'+'|'-')(digit+);
\end{rail}

\begin{rail}
	digit : "0--9";
\end{rail}

\bigskip

Like any other \faust expression, numbers are signal processors. For example the number $0.95$ is a signal processor of type $\mathbb{S}^{0}\rightarrow\mathbb{S}^{1}$ that transforms an empty tuple of signals $()$ into a 1-tuple of signals $(y)$ such that $\forall t\in\mathbb{N}, y(t)=0.95$.

Operations on \textit{integer} numbers follow the standard C semantics for \lstinline'+, -, *' operations and can overflow if the result cannot be represented as a 32-bit integer. The \lstinline'/' operation is treated separately and casts both of its arguments to floats before performing the division, and thus the result takes the float type.

%\begin{tabular}{|l|l|l|}
%\hline
%\textbf{Syntax} & \textbf{Type}  & \textbf{Description} \\
%\hline
%$n$ & $\mathbb{S}^{0}\rightarrow\mathbb{S}^{1}$ & integer number: $y(t)=n$ \\
%$r$ & $\mathbb{S}^{0}\rightarrow\mathbb{S}^{1}$ & floating point number: $y(t)=r$ \\
%\hline

%\end{tabular}

%--------------------------------------------------------------------------------------------------------------
\subsection{Route Primitive }
%--------------------------------------------------------------------------------------------------------------

The  \lstinline'route' primitive facilitates the routing of signals in \faust. It has the following syntax:

\begin{lstlisting}
route(A,B,a,b,c,d,...)
route(A,B,(a,b),(c,d),...)
\end{lstlisting}

where:

\begin{itemize}
	\item \lstinline'A' is the number of input signals, as an integer \textit{constant numerical expression}, automatically promoted to \textit{int}
	\item \lstinline'B' is the number of output signals, as an integer \textit{constant numerical expression}, automatically promoted to \textit{int}
	\item \lstinline'a,b' or \lstinline'(a,b)' is an input/output pair, as integers \textit{constant numerical expressions}, automatically promoted to \textit{int}
\end{itemize}

Inputs are numbered from \lstinline'1' to \lstinline'A' and outputs are numbered from \lstinline'1' to \lstinline'B'. There can be any number of input/output pairs after the declaration of \lstinline'A' and \lstinline'B'.

For example, crossing two signals can be carried out with:

\begin{lstlisting}
process = route(2,2,1,2,2,1);
\end{lstlisting}

In that case, \lstinline'route' has 2 inputs and 2 outputs. The first input (1) is connected to the second output (2) and the second input (2) is connected to the first output (1).

Note that parentheses can optionally be used to define a pair, so the previous example can also be written as:

\begin{lstlisting}
process = route(2,2,(1,2),(2,1));
\end{lstlisting}

More complex expressions can be written using algorithmic constructions, like the following one to cross N signals:

\begin{lstlisting}
// cross 10 signals: 
// input 0 -> output 10, 
// input 1 -> output 9, 
// ..., 
// input 9 -> output 0

N = 10;
r = route(N,N,par(i,N,(i+1,N-i)));

process = r;
\end{lstlisting}

%--------------------------------------------------------------------------------------------------------------
\subsection{Waveform Primitive}
%--------------------------------------------------------------------------------------------------------------

A waveform is a fixed periodic signal defined by a list of samples as literal numbers. A waveform has two outputs. The first output is constant and indicates the size (number of samples) of the period. The second output is the periodic signal itself.

\begin{rail}
	waveform : "waveform" lbrace (number + ',') rbrace;
\end{rail}

For example \lstinline'waveform {0,1,2,3}' produces two outputs, the constant signal 4 and the periodic signal \lstinline'0,1,2,3,0,1,2,3,0,1'\ldots.

Please note that \lstinline'waveform' works nicely with \lstinline'rdtable'. Its first output, known at compile time, gives the size of the table, while the second signal gives the content of the table. Here is an example:
\begin{lstlisting}
process = waveform {10,20,30,40,50,60,70}, %(7)~+(3) : rdtable;
\end{lstlisting}

\bigskip

%--------------------------------------------------------------------------------------------------------------
\subsection{Soundfile Primitive}
%--------------------------------------------------------------------------------------------------------------

The \lstinline'soundfile("label[url:\{'path1';'path2';'path3';\}]", n)' primitive allows access to a list of externally defined sound resources, described as a label followed by the list of their filenames or complete paths (possibly using the '\texttt{\%i}' syntax, as in the label part). The simplified syntax \lstinline'soundfile("label[url:path]", n)' or \lstinline'soundfile("label", n)' (where the label is used as the soundfile path) allows a single file to be used. All sound resources are concatenated in a single data structure, and each item can be accessed and used independently.

A soundfile has:

\begin{itemize}
	\item two inputs: the sound number (as an integer between 0 and 255, automatically promoted to \textit{int}), and the read index in the sound (automatically promoted to \textit{int}, which will access the last sample of the sound if the read index is greater than the sound length)
	\item two fixed outputs: the first one is the length in samples of the currently accessed sound, the second one is the nominal sample rate in Hz of the currently accessed sound
	\item \lstinline'n' additional outputs for the sound channels themselves, as an integer \textit{constant numerical expression}
\end{itemize}

If more outputs than the actual number of channels in the soundfile are used, the sound channels will be automatically duplicated up to the wanted number of outputs (so for instance if a stereo sound is used with four output channels, the same group of two channels will be duplicated).

If the soundfile cannot be loaded for whatever reason, a default sound with one channel, a length of 1024 frames and null outputs (with samples of value 0) will be used. Note also that soundfiles are entirely loaded in memory by the architecture file, so that the read index signal can access any sample.

Specialized architecture files are responsible for loading the actual soundfile. The \lstinline'SoundUI' C++ class located in the \lstinline'faust/gui/SoundUI.h' file implements the \lstinline'void addSoundfile(label, url, sf_zone)' method, which loads the actual soundfiles using the {\it libsndfile} library, or possibly specific audio file loading code (in the case of the JUCE framework, for instance), and sets up the \lstinline'sf_zone' sound memory pointers.

Note that a specific architecture file can choose to access and use sound resources created by other means (that is, not directly loaded from a soundfile). For instance, a mapping between labels and sound resources defined in memory could be used, with additional code responsible for actually setting up all sound memory pointers when \lstinline'void addSoundfile(label, url, sf_zone)' is called by the \lstinline'buildUserInterface' mechanism.

%--------------------------------------------------------------------------------------------------------------
\subsection{C-equivalent primitives}
%--------------------------------------------------------------------------------------------------------------

Most \faust primitives are analogous to their C counterparts but adapted to signal processing.
For example \lstinline|+| is a function of type $\mathbb{S}^{2}\rightarrow\mathbb{S}^{1}$ that transforms a pair of signals $(x_1,x_2)$ into a 1-tuple of signals $(y)$ such that $\forall t\in\mathbb{N}, y(t)=x_{1}(t)+x_{2}(t)$. The function \lstinline|-| has type $\mathbb{S}^{2}\rightarrow\mathbb{S}^{1}$ and transforms a pair of signals $(x_1,x_2)$ into a 1-tuple of signals $(y)$ such that $\forall t\in\mathbb{N}, y(t)=x_{1}(t)-x_{2}(t)$.

Please be aware that the unary \lstinline|-| exists only in a limited form\marginpar{Warning: unlike other programming languages the unary operator \lstinline|-| only exists in limited form in \faust}. It can be used with numbers (\lstinline|-0.5|) and variables (\lstinline|-myvar|), but not with expressions surrounded by parentheses, because in this case it represents a partial application.  For instance \lstinline|-(a * b)| is a partial application. It is syntactic sugar for \lstinline|_,(a * b) : -|. If you want to negate a complex term in parentheses, you'll have to use \lstinline|0 - (a * b)| instead.

The primitives may use the \texttt{int} type for their arguments, but will automatically use the \texttt{float} type when the actual computation requires it. For instance \lstinline|1/2| using \texttt{int} type arguments will correctly result in \lstinline|0.5| in \texttt{float} type. Logical and shift primitives use the \texttt{int} type.

\bigskip

\begin{tabular}{|l|l|l|}
	\hline
	\textbf{Syntax}      & \textbf{Type}                             & \textbf{Description}                                     \\
	\hline
	$n$                  & $\mathbb{S}^{0}\rightarrow\mathbb{S}^{1}$ & integer number: $y(t)=n$                                 \\
	$n.m$                & $\mathbb{S}^{0}\rightarrow\mathbb{S}^{1}$ & floating point number: $y(t)=n.m$                        \\

	\texttt{\_}          & $\mathbb{S}^{1}\rightarrow\mathbb{S}^{1}$ & identity function: $y(t)=x(t)$                           \\
	\texttt{!}           & $\mathbb{S}^{1}\rightarrow\mathbb{S}^{0}$ & cut function: $\forall x\in\mathbb{S},(x)\rightarrow ()$ \\

	\texttt{int}         & $\mathbb{S}^{1}\rightarrow\mathbb{S}^{1}$ & cast into an int signal: $y(t)=(int)x(t)$                \\
	\texttt{float}       & $\mathbb{S}^{1}\rightarrow\mathbb{S}^{1}$ & cast into a float signal: $y(t)=(float)x(t)$             \\

	\texttt{+}           & $\mathbb{S}^{2}\rightarrow\mathbb{S}^{1}$ & addition: $y(t)=x_{1}(t)+x_{2}(t)$                       \\
	\texttt{-}           & $\mathbb{S}^{2}\rightarrow\mathbb{S}^{1}$ & subtraction: $y(t)=x_{1}(t)-x_{2}(t)$                    \\
	\texttt{*}           & $\mathbb{S}^{2}\rightarrow\mathbb{S}^{1}$ & multiplication: $y(t)=x_{1}(t)*x_{2}(t)$                 \\
	\texttt{$\land$}     & $\mathbb{S}^{2}\rightarrow\mathbb{S}^{1}$ & power: $y(t)=x_{1}(t)^{x_{2}(t)}$                        \\
	\texttt{/}           & $\mathbb{S}^{2}\rightarrow\mathbb{S}^{1}$ & division: $y(t)=x_{1}(t)/x_{2}(t)$                       \\
	\texttt{\%}          & $\mathbb{S}^{2}\rightarrow\mathbb{S}^{1}$ & modulo: $y(t)=x_{1}(t)\%x_{2}(t)$                        \\

	\texttt{\&}          & $\mathbb{S}^{2}\rightarrow\mathbb{S}^{1}$ & bitwise AND: $y(t)=x_{1}(t)\&x_{2}(t)$                   \\
	\texttt{|}           & $\mathbb{S}^{2}\rightarrow\mathbb{S}^{1}$ & bitwise OR: $y(t)=x_{1}(t)|x_{2}(t)$                     \\
	\texttt{xor}         & $\mathbb{S}^{2}\rightarrow\mathbb{S}^{1}$ & bitwise XOR: $y(t)=x_{1}(t)\land x_{2}(t)$               \\

	\texttt{<}\texttt{<} & $\mathbb{S}^{2}\rightarrow\mathbb{S}^{1}$ & arithmetic shift left: $y(t)=x_{1}(t) << x_{2}(t)$       \\
	\texttt{>}\texttt{>} & $\mathbb{S}^{2}\rightarrow\mathbb{S}^{1}$ & arithmetic shift right: $y(t)=x_{1}(t) >> x_{2}(t)$      \\


	\texttt{<}           & $\mathbb{S}^{2}\rightarrow\mathbb{S}^{1}$ & less than: $y(t)=x_{1}(t) < x_{2}(t)$                    \\
	\texttt{<=}          & $\mathbb{S}^{2}\rightarrow\mathbb{S}^{1}$ & less than or equal: $y(t)=x_{1}(t) <= x_{2}(t)$          \\
	\texttt{>}           & $\mathbb{S}^{2}\rightarrow\mathbb{S}^{1}$ & greater than: $y(t)=x_{1}(t) > x_{2}(t)$                 \\
	\texttt{>=}          & $\mathbb{S}^{2}\rightarrow\mathbb{S}^{1}$ & greater than or equal: $y(t)=x_{1}(t) >= x_{2}(t)$       \\
	\texttt{==}          & $\mathbb{S}^{2}\rightarrow\mathbb{S}^{1}$ & equal: $y(t)=x_{1}(t) == x_{2}(t)$                       \\
	\texttt{!=}          & $\mathbb{S}^{2}\rightarrow\mathbb{S}^{1}$ & not equal: $y(t)=x_{1}(t) != x_{2}(t)$                   \\

	\hline
\end{tabular}

\bigskip

%--------------------------------------------------------------------------------------------------------------
\subsection{\texttt{math.h}-equivalent primitives}
%--------------------------------------------------------------------------------------------------------------

Most of the C \texttt{math.h} functions are also built-in as primitives (the others are defined as external functions in file \texttt{maths.lib}).
The primitives may use the \texttt{int} type for their arguments, but will automatically use the \texttt{float} type when the actual computation requires it.

\bigskip

\begin{tabular}{|l|l|l|}

	\hline
	\textbf{Syntax}    & \textbf{Type}                             & \textbf{Description}                                                                    \\
	\hline

	\texttt{acos}      & $\mathbb{S}^{1}\rightarrow\mathbb{S}^{1}$ & arc cosine: $y(t)=\mathrm{acosf}(x(t))$                                                 \\
	\texttt{asin}      & $\mathbb{S}^{1}\rightarrow\mathbb{S}^{1}$ & arc sine: $y(t)=\mathrm{asinf}(x(t))$                                                   \\
	\texttt{atan}      & $\mathbb{S}^{1}\rightarrow\mathbb{S}^{1}$ & arc tangent: $y(t)=\mathrm{atanf}(x(t))$                                                \\
	\texttt{atan2}     & $\mathbb{S}^{2}\rightarrow\mathbb{S}^{1}$ & arc tangent: $y(t)=\mathrm{atan2f}(x_{1}(t), x_{2}(t))$                                 \\

	\texttt{cos}       & $\mathbb{S}^{1}\rightarrow\mathbb{S}^{1}$ & cosine: $y(t)=\mathrm{cosf}(x(t))$                                                      \\
	\texttt{sin}       & $\mathbb{S}^{1}\rightarrow\mathbb{S}^{1}$ & sine: $y(t)=\mathrm{sinf}(x(t))$                                                        \\
	\texttt{tan}       & $\mathbb{S}^{1}\rightarrow\mathbb{S}^{1}$ & tangent: $y(t)=\mathrm{tanf}(x(t))$                                                     \\

	\texttt{exp}       & $\mathbb{S}^{1}\rightarrow\mathbb{S}^{1}$ & base-e exponential: $y(t)=\mathrm{expf}(x(t))$                                          \\
	\texttt{log}       & $\mathbb{S}^{1}\rightarrow\mathbb{S}^{1}$ & base-e logarithm: $y(t)=\mathrm{logf}(x(t))$                                            \\
	\texttt{log10}     & $\mathbb{S}^{1}\rightarrow\mathbb{S}^{1}$ & base-10 logarithm: $y(t)=\mathrm{log10f}(x(t))$                                         \\
	\texttt{pow}       & $\mathbb{S}^{2}\rightarrow\mathbb{S}^{1}$ & power: $y(t)=\mathrm{powf}(x_{1}(t),x_{2}(t))$                                          \\
	\texttt{sqrt}      & $\mathbb{S}^{1}\rightarrow\mathbb{S}^{1}$ & square root: $y(t)=\mathrm{sqrtf}(x(t))$                                                \\
	\texttt{abs}       & $\mathbb{S}^{1}\rightarrow\mathbb{S}^{1}$ & absolute value (int): $y(t)=\mathrm{abs}(x(t))$                                         \\
	                   &                                           & absolute value (float): $y(t)=\mathrm{fabsf}(x(t))$                                     \\
	\texttt{min}       & $\mathbb{S}^{2}\rightarrow\mathbb{S}^{1}$ & minimum: $y(t)=\mathrm{min}(x_{1}(t),x_{2}(t))$                                         \\
	\texttt{max}       & $\mathbb{S}^{2}\rightarrow\mathbb{S}^{1}$ & maximum: $y(t)=\mathrm{max}(x_{1}(t),x_{2}(t))$                                         \\
	\texttt{fmod}      & $\mathbb{S}^{2}\rightarrow\mathbb{S}^{1}$ & float modulo: $y(t)=\mathrm{fmodf}(x_{1}(t),x_{2}(t))$                                  \\
	\texttt{remainder} & $\mathbb{S}^{2}\rightarrow\mathbb{S}^{1}$ & float remainder: $y(t)=\mathrm{remainderf}(x_{1}(t),x_{2}(t))$                          \\

	\texttt{floor}     & $\mathbb{S}^{1}\rightarrow\mathbb{S}^{1}$ & largest int $\leq$: $y(t)=\mathrm{floorf}(x(t))$                                        \\
	\texttt{ceil}      & $\mathbb{S}^{1}\rightarrow\mathbb{S}^{1}$ & smallest int $\geq$: $y(t)=\mathrm{ceilf}(x(t))$                                        \\
	\texttt{rint}      & $\mathbb{S}^{1}\rightarrow\mathbb{S}^{1}$ & closest int using the current rounding mode: $y(t)=\mathrm{rintf}(x(t))$                \\
	\texttt{round}     & $\mathbb{S}^{1}\rightarrow\mathbb{S}^{1}$ & nearest int value, regardless of the current rounding mode: $y(t)=\mathrm{rintf}(x(t))$ \\

	\hline
\end{tabular}
\bigskip

%--------------------------------------------------------------------------------------------------------------
\subsection{Delay, Table, Selector primitives}
%--------------------------------------------------------------------------------------------------------------

The following primitives allow to define delays, read-only and read-write tables and 2 or 3-ways selectors (see figure \ref{fig-delays}), and some other specific primitives.

\bigskip
\begin{tabular}{|l|l|l|}
	\hline
	\textbf{Syntax}  & \textbf{Type}                             & \textbf{Description}                                                         \\
	\hline

	\texttt{mem}     & $\mathbb{S}^{1}\rightarrow\mathbb{S}^{1}$ & 1-sample delay: 	$y(t+1)=x(t),y(0)=0$                                        \\
	\texttt{prefix}  & $\mathbb{S}^{2}\rightarrow\mathbb{S}^{1}$ & 1-sample delay:  	$y(t+1)=x_{2}(t),y(0)=x_{1}(0)$                            \\
	\texttt{@}       & $\mathbb{S}^{2}\rightarrow\mathbb{S}^{1}$ & variable delay:  	$y(t)=x_{1}(t-x_{2}(t)), x_{1}(t<0)=0$                     \\

	\texttt{rdtable} & $\mathbb{S}^{3}\rightarrow\mathbb{S}^{1}$ & read-only table:	$y(t)=T[r(t)]$                                              \\
	\texttt{rwtable} & $\mathbb{S}^{5}\rightarrow\mathbb{S}^{1}$ & read-write table:	$T[w(t)]=c(t); y(t)=T[r(t)]$                               \\

	\texttt{select2} & $\mathbb{S}^{3}\rightarrow\mathbb{S}^{1}$ & select between 2 signals:	$T[]=\{x_{0}(t),x_{1}(t)\}; y(t)=T[s(t)]$          \\
	\texttt{select3} & $\mathbb{S}^{4}\rightarrow\mathbb{S}^{1}$ & select between 3 signals:	$T[]=\{x_{0}(t),x_{1}(t),x_{2}(t)\}; y(t)=T[s(t)]$ \\

	\texttt{lowest}  & $\mathbb{S}^{1}\rightarrow\mathbb{S}^{1}$ & $y(t)=\min_{j=0}^{\inf} x(j)$                                                \\
	\texttt{highest} & $\mathbb{S}^{1}\rightarrow\mathbb{S}^{1}$ & $y(t)=\max_{j=0}^{\inf} x(j)$                                                \\


	\hline
\end{tabular}
\bigskip

\begin{figure}
	\centering
	\includegraphics[scale=0.6]{illustrations/faust-diagram4}
	\includegraphics[scale=0.6]{illustrations/faust-diagram5}
	\includegraphics[scale=0.6]{illustrations/faust-diagram6}
	\caption{Delays, tables and selectors primitives }
	\label{fig-delays}
\end{figure}

The size input of \textit{rdtable} and \textit{rwtable} are integer \textit{constant numerical expressions} automatically promoted to \textit{int}, and the read and write indexes are also automatically promoted to \textit{int}. The delay value is automatically promoted to \textit{int}.

%--------------------------------------------------------------------------------------------------------------
\subsection{User Interface Elements}
%--------------------------------------------------------------------------------------------------------------

\faust user interface widgets allow an abstract description of the user interface from within the \faust code. This description is
independent of any GUI toolkits. It is based on \emph{buttons}, \emph{checkboxes}, \emph{sliders}, etc. that are grouped together
vertically and horizontally using appropriate grouping schemes.

All these GUI elements produce signals. A button for example (see figure \ref{fig-button}) produces a signal which is 1 when the button is pressed and 0 otherwise. These signals can be freely combined with other audio signals.

\begin{figure}[h]
	\centering
	\includegraphics[scale=0.5]{illustrations/button}
	\caption{User Interface Button}
	\label{fig-button}
\end{figure}

\bigskip

\begin{tabular}{|l|l|}
	\hline
	\textbf{Syntax}                                                           & \textbf{Example}                      \\
	\hline
	\texttt{button(\farg{str})}                                               & \texttt{button("play")}               \\
	\texttt{checkbox(\farg{str})}                                             & \texttt{checkbox("mute")}             \\
	\texttt{vslider(\farg{str},\farg{cur},\farg{min},\farg{max},\farg{step})} & \texttt{vslider("vol",50,0,100,1)}    \\
	\texttt{hslider(\farg{str},\farg{cur},\farg{min},\farg{max},\farg{step})} & \texttt{hslider("vol",0.5,0,1,0.01)}  \\
	\texttt{nentry(\farg{str},\farg{cur},\farg{min},\farg{max},\farg{step})}  & \texttt{nentry("freq",440,0,8000,1)}  \\
	\texttt{vgroup(\farg{str},\farg{block-diagram})}                          & \texttt{vgroup("reverb", \ldots)}     \\
	\texttt{hgroup(\farg{str},\farg{block-diagram})}                          & \texttt{hgroup("mixer", \ldots)}      \\
	\texttt{tgroup(\farg{str},\farg{block-diagram})}                          & \texttt{tgroup("parametric", \ldots)} \\
	\texttt{vbargraph(\farg{str},\farg{min},\farg{max})}                      & \texttt{vbargraph("input",0,100)}     \\
	\texttt{hbargraph(\farg{str},\farg{min},\farg{max})}                      & \texttt{hbargraph("signal",0,1.0)}    \\
	\texttt{attach}                                                           & \texttt{attach(x, vumeter(x))}        \\
	\hline
\end{tabular}

All numerical parameters (like {\it cur}, {\it min}, {\it max}, {\it step}) are \textit{constant numerical expressions}.

\bigskip
\subsubsection{Labels}
Every user interface widget has a label (a string) that identifies it and informs the user of its purpose. There are three important mechanisms associated with labels (and coded inside the string): \textit{variable parts}, \textit{pathnames} and \textit{metadata}.

\paragraph{Variable parts.}
Labels can contain variable parts. These variable parts are indicated by the sign '\texttt{\%}' followed by the name of a variable. During compilation each label is processed in order to replace the variable parts by the value of the variable.
For example \lstinline'par(i,8,hslider("Voice %i", 0.9, 0, 1, 0.01))' creates 8 different sliders in parallel :

\begin{lstlisting}
hslider("Voice 0", 0.9, 0, 1, 0.01),
hslider("Voice 1", 0.9, 0, 1, 0.01),
...
hslider("Voice 7", 0.9, 0, 1, 0.01).
\end{lstlisting}

while \lstinline'par(i,8,hslider("Voice", 0.9, 0, 1, 0.01))' would have created only one slider and duplicated its output 8 times.

The variable part can have an optional format digit.
For example \lstinline'"Voice %2i"' would indicate to use two digits when inserting the value of i in the string.

An escape mechanism is provided.
If the sign \lstinline'%' is followed by itself, it will be included in the resulting string.
For example \lstinline'"feedback (%%)"' will result in \lstinline'"feedback (%)"'.

The variable name can be enclosed in curly brackets to clearly separate it from the rest of the string, as in \lstinline'par(i,8,hslider("Voice %{i}", 0.9, 0, 1, 0.01))'.



\paragraph{Pathnames.}
Thanks to horizontal, vertical and tabs groups, user interfaces have a hierarchical structure analog to a hierarchical file system. Each widget has an associated \textit{pathname} obtained by concatenating the labels of all its surrounding groups with its own label.

In the following example:
\begin{lstlisting}
hgroup("Foo",
	...
	vgroup("Faa", 
		...
		hslider("volume",...)
		...
	)
	...
)
\end{lstlisting}
the volume slider has pathname \lstinline'/h:Foo/v:Faa/volume'.

In order to give more flexibility to the design of user interfaces, it is possible to explicitly specify the absolute or relative pathname of a widget directly in its label.

In our previous example the pathname of:
\begin{lstlisting}
	hslider("../volume",...)
\end{lstlisting}
would have been \lstinline'"/h:Foo/volume"', while the pathname of:
\begin{lstlisting}
	hslider("t:Fii/volume",...)
\end{lstlisting}
would have been:
\lstinline'"/h:Foo/v:Faa/t:Fii/volume"'.

The grammar for labels with pathnames is the following:
% \begin{grammar}
%   <label> ::= 
%   \begin{syntdiag}
% 	<path> <name>
%   \end{syntdiag}
% \end{grammar}
% %
% \begin{grammar}
%  <path> ::= 
%   \begin{syntdiag}
% 	\begin{stack} \\ "/" \end{stack} 
% 	\begin{stack} \\ \begin{rep} <folder> "/" \end{rep} \end{stack} 
%  \end{syntdiag}
% \end{grammar}
% %
% \begin{grammar}
%  <folder> ::= 
%   \begin{syntdiag}
% 	\begin{stack}
% 		".." \\ 
% 		\begin{stack} "h:" \\ "v:" \\ "t:" \end{stack} <name>
% 	\end{stack}
%  \end{syntdiag}
% \end{grammar}

\begin{rail}
	label : path name;
\end{rail}

\begin{rail}
	path : (| '/') (| (folder '/')+);
\end{rail}

\begin{rail}
	folder : (".." | ("h:" | "v:" | "t:" ) name);
\end{rail}

\paragraph{Metadata}
Widget labels can contain metadata enclosed in square brackets. These metadata associate a key with a value and are used to provide additional information to the architecture file.  They are typically used to improve the look and feel of the user interface.
The \faust code:
\begin{lstlisting}
process = *(hslider("foo [key1: val 1][key2: val 2]", 
					0, 0, 1, 0.1));
\end{lstlisting}

will produce and the corresponding C++ code:

\begin{lstlisting}
class mydsp : public dsp {
	...
	virtual void buildUserInterface(UI* interface) {
	  interface->openVerticalBox("m");
	  interface->declare(&fslider0, "key1", "val 1");
	  interface->declare(&fslider0, "key2", "val 2");
	  interface->addHorizontalSlider("foo", 
	  	&fslider0, 0.0f, 0.0f, 1.0f, 0.1f);
	  interface->closeBox();
	}
...
};
\end{lstlisting}

All the metadata are removed from the label by the compiler and
transformed in calls to the \lstinline'UI::declare()' method. All these
\lstinline'UI::declare()' calls will always take place before the \lstinline'UI::AddSomething()'
call that creates the User Interface element. This allows the
\lstinline'UI::AddSomething()'  method to make full use of the available metadata.

It is the role of the architecture file to decide what to do with these
metadata. The \lstinline'jack-qt.cpp' architecture file for example implements the
following:
\begin{enumerate}
	\item \lstinline'"...[style:knob]..."' creates a rotating knob instead of a regular
	      slider or nentry.
	\item \lstinline'"...[style:led]..."' in a bargraph's label creates a small LED instead
	      of a full bargraph
	\item \lstinline'"...[unit:dB]..."' in a bargraph's label creates a more realistic
	      bargraph with colors ranging from green to red depending of the level of
	      the value
	\item \lstinline'"...[unit:xx]..."' in a widget postfixes the value displayed with xx
	\item \lstinline'"...[tooltip:bla bla]..."' add a tooltip to the widget
	\item \lstinline'"...[osc:/address min max]..."' Open Sound Control message alias
\end{enumerate}

Moreover starting a label with a number option like in \lstinline'"[1]..."' provides
a convenient means to control the alphabetical order of the widgets.

\subsubsection{Attach}
The \lstinline'attach' primitive takes two input signals and produce one output signal which is a copy of the first input. The role of \lstinline'attach' is to force its second input signal to be compiled with the first one. From a mathematical point of view \lstinline'attach(x,y)' is equivalent to \lstinline'1*x+0*y', which is in turn equivalent to \lstinline'x', but it tells the compiler not to optimize-out \lstinline'y'.

To illustrate this role let say that we want to develop a mixer application with a vumeter for each input signals. Such vumeters can be easily coded in \faust using an envelop detector connected to a bargraph. The problem is that these envelop signals have no role in the output signals. Using \lstinline'attach(x,vumeter(x))' one can tell the compiler that when \lstinline'x' is compiled \lstinline'vumeter(x)' should also be compiled.
