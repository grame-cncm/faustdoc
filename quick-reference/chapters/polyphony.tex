%---------------------------------------------------
\chapter{Polyphonic support} 
\label{poly}
%---------------------------------------------------

Directly programming polyphonic instruments in \faust is perfectly possible. It is also necessary when very complex signal interactions between the different voices have to be described\footnote{Like sympathetic string resonance in a physical model of a piano, for instance.}.

But since all voices would always be computed, this approach could be too CPU-costly for simpler or more limited needs. In this case, describing a single voice in a \faust DSP program and externally combining several of them with a special  {\it polyphonic instrument aware} architecture file is a better solution. Moreover, this special architecture file takes care of dynamic voice allocation and MIDI message decoding and mapping.

\section{Polyphony-ready DSP code}

By convention \faust architecture files with polyphonic capabilities expect to find control parameters named {\it freq}, {\it gain}, and {\it gate}.  Metadata such as \code{declare nvoices "8";} can be added to the source code to specify a desired number of voices.

In the case of MIDI control, the {\it freq} parameter (which should be a frequency) will be automatically computed from MIDI note numbers, {\it gain} (which should be a value between 0 and 1) from velocity, and {\it gate} from {\it keyon/keyoff} events. Thus, gate can be used as a trigger signal for any envelope generator, etc.

\section{Using the mydsp\_poly class}

A single voice has to be described by a \faust DSP program, and the \code{mydsp_poly} class is then used to combine several voices and create a polyphony-ready DSP:

\begin{itemize}

\item  the {\it faust/dsp/poly-dsp.h} file contains the definition of the \code{mydsp_poly} class used to wrap the DSP voice into the polyphonic architecture. This class maintains an array of \code{dsp}-type objects, manages dynamic voice allocation, decodes and maps MIDI messages, mixes all running voices, and stops a voice when its output level decreases below a given threshold.

\item as a subclass of DSP, the \code{mydsp_poly}  class redefines the \code{buildUserInterface} method. By convention all allocated voices are grouped in a global  {\it Polyphonic} tab group. The first tab contains a {\it Voices} group, a master-like component used to change parameters on all voices at the same time, with a {\it Panic} button to stop running voices\footnote{An internal control grouping mechanism has been defined to automatically dispatch a user interface action done on the master component to all linked voices, except for the  {\it freq}, {\it gain}, and {\it gate} controls.}, followed by one tab for each voice. Graphical user interface components will then reflect the multi-voice structure of the new polyphonic DSP (Figure \ref{fig:poly-ui}). 

 \end{itemize}
 
\begin{figure}[!ht]
\begin{center}
\includegraphics[width=0.6\columnwidth]{images/poly_ui}
\caption{\footnotesize Extended multi-voices GUI interface}
\label{fig:poly-ui}
\end{center}
\end{figure}

The resulting polyphonic DSP object can be used as usual, connected with the needed audio driver, and possibly other UI control objects like OSCUI, httpdUI, etc. Having this new UI hierarchical view allows complete OSC control of each single voice and their control parameters, but also all voices using the master component. 

The following OSC messages reflect the same DSP code either compiled normally,  or in polyphonic mode (only part of the OSC hierarchies are displayed here):

\footnotesize
\begin{lstlisting}

// Mono mode

/0x00/0x00/vol f -10.0
/0x00/0x00/pan f 0.0

// Polyphonic mode

/Polyphonic/Voices/0x00/0x00/pan f 0.0
/Polyphonic/Voices/0x00/0x00/vol f -10.0
...
/Polyphonic/Voice1/0x00/0x00/vol f -10.0
/Polyphonic/Voice1/0x00/0x00/pan f 0.0
...
/Polyphonic/Voice2/0x00/0x00/vol f -10.0
/Polyphonic/Voice2/0x00/0x00/pan f 0.0
...
\end{lstlisting}
\normalsize

The polyphonic instrument allocator takes the DSP to be used for one voice\footnote{The DSP object will be automatically cloned in the mydsp\_poly class to create all needed voices.}, the desired number of voices, the {\it dynamic voice allocation} state\footnote{Voices may always be running, or dynamically started/stopped in case of MIDI control.}, and the {\it group} state, which controls whether separate voices are displayed (Figure \ref{fig:poly-ui}): 

\footnotesize
\begin{lstlisting}
    DSP = new mydsp_poly(dsp, 2, true, true);  
\end{lstlisting}
    
\normalsize
With the following code, note that a polyphonic instrument may be used outside of a MIDI control context, so that all voices will always be running and can be controlled with OSC messages, for instance:

\footnotesize
\begin{lstlisting}
    DSP = new mydsp_poly(dsp, 8, false, true);
\end{lstlisting}

\normalsize
    
\section{Controlling the polyphonic instrument}

The \code{mydsp_poly} class is also ready for MIDI control and can react to {\it keyon/keyoff} and {\it pitchwheel} messages. Other MIDI control parameters can directly be added in the DSP source code. 

\section{Deploying the polyphonic instrument}

Several architecture files and associated scripts have been updated to handle polyphonic instruments:

As an example on macOS, the script \code{faust2caqt foo.dsp} can be used to create a polyphonic CoreAudio/Qt application. The desired number of voices is either declared via \code{nvoices} metadata or changed with the \code{-nvoices num} additional parameter\footnote{The \code{-nvoices} parameter takes precedence over the metadata value.}. MIDI control is activated using the \code{-midi} parameter. 

The number of allocated voices can possibly be changed at runtime using the \code{-nvoices} parameter to change the default value (so using \code{./foo -nvoices 16} for instance). 

Several other scripts have been adapted using the same conventions.

\section{Polyphonic instrument with a global output effect}

Polyphonic instruments may be used with an output effect. Putting that effect in the main \faust code is not a good idea since it would be instantiated for each voice which would be very inefficient. This is a typical use case for the  \code{dsp_sequencer} class previously presented with the polyphonic DSP connected in sequence with a unique global effect (Figure \ref{fig:poly-ui-effect}). 

\code{faustcaqt inst.dsp -effect effect.dsp} with inst.dsp and effect.dsp in the same folder,  and the number of outputs of the instrument matching the number of inputs of the effect, has to be used. A \code{dsp_sequencer} object will be created to combine the polyphonic instrument in sequence with the single output effect. 

Polyphony-ready {\it faust2xx} scripts will then compile the polyphonic instrument and the effect, combine them in sequence, and create a ready-to-use DSP.  

\begin{figure}[!ht]
\begin{center}
\includegraphics[width=0.48\columnwidth]{images/poly_ui_effect1}
\includegraphics[width=0.48\columnwidth]{images/poly_ui_effect2}
\caption{\footnotesize Polyphonic instrument with output effect GUI interface: left tab window shows the polyphonic instrument with its {\it Voices} group only, right tab window shows the output effect.}
\label{fig:poly-ui-effect}
\end{center}
\end{figure}

\subsection{Integrated global output effect}

Starting with the 2.5.17 version, a new convention has been defined to directly integrate a global output effect inside the DSP source code itself. The effect has simply to be declared in a \code{effect =  effect_code;} line in the source. Here is a more complete source code example:

\begin{lstlisting}
import("stdfaust.lib");
process = pm.clarinet_ui_MIDI <: _,_;
effect = dm.freeverb_demo;
\end{lstlisting}

The architecture script then separates the instrument description itself (the \code{process = ...} definition) from the effect definition (the  \code{effect = ...} definition), possibly adapts the instrument number of outputs to the effect number of inputs, compiles each part separately, and combines them with the \code{dsp_sequencer} object.

A new \code{auto} parameter has been defined for {\it faust2xx} scripts, as in the \code{faustcaqt inst.dsp -effect auto} command, for example.

\subsection{Integrated global output effect and libfaust}

For developers using the libfaust library, a helper file named \code{faust/dsp/poly-dsp-tools.h} is available. It defines an API to automatically create a polyphonic instrument with an output effect, starting from a DSP source file using the  \code{effect = ...} convention. The function \code{createPolyDSPFactoryFromString} or  \\ \code{createPolyDSPFactoryFromFile} must be used to create the polyphonic DSP factory. Next, the \code{createPolyDSPInstance} function creates the polyphonic object (a subclass of the \code{dsp_poly} type) to be used like a regular \code{dsp} type object. 

After the DSP factory has been compiled, your application or plugin may want to save/restore it in order to save \faust to LLVM IR compilation or even JIT compilation time at next use. To get the internal factory compiled code, several functions are available:

\begin{itemize}
\item \code{writePolyDSPFactoryToIRFile} allows you to save the polyphonic factory LLVM IR (in textual format) in a file,
\item \code{writePolyDSPFactoryToBitcodeFile} allows you to save the polyphonic factory LLVM IR (in binary format) in a file,
\item \code{writePolyDSPFactoryToMachineFile} allows you to save the polyphonic factory executable machine code in a file.
\end{itemize}

To re-create a DSP factory from a previously saved code, several functions are available:

\begin{itemize}
\item \code{readPolyDSPFactoryFromIRFile} allows you to create a polyphonic DSP factory from a file containing the LLVM IR (in textual format),
\item \code{readPolyDSPFactoryFromBitcodeFile} allows you to create a polyphonic  factory from a file containing the LLVM IR (in binary format),
\item \code{readPolyDSPFactoryFromMachineFile} allows you to create a polyphonic DSP factory from a file containing the executable machine code.
\end{itemize}
