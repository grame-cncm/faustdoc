%---------------------------------------------------
\chapter{HTTP support} 
\label{http}
%---------------------------------------------------

Like OSC, several \faust architectures also provide HTTP support. This allows \faust applications to be remotely controlled from any web browser using specific URLs. Moreover, OSC and HTTPD can be freely combined.

While OSC support is installed by default when \faust is built, this is not the case for HTTP. That's because it depends on the GNU \emph{libmicrohttpd} library, which is usually not installed by default on the system. An additional \lstinline'make httpd' step is therefore required when compiling and installing \faust:
\begin{lstlisting}
make httpd
make
sudo make install
\end{lstlisting}
Note that \lstinline'make httpd' will fail if \emph{libmicrohttpd} is not available on the system.

HTTP support can be activated using the \code{-httpd} option when building the audio application with the appropriate \code{faust2xxx} command. The following table (Table \ref{tab:httparch}) lists the \faust architectures that provide HTTP support. 

\begin{table}[htp]
\begin{center}
\begin{tabular}{rcc}
\hline
\bf{Audio system} 	& \bf{Environment} & \bf{HTTP support}	\\
\hline
%\OSTab{Linux} \\
%\multicolumn{3}{|l|}{Linux} \\
%\hline
\\
\emph{Linux}\\
Alsa  			& GTK, Qt, Console		& yes\\
Jack 			& GTK, Qt, Console		& yes\\
Netjack 			& GTK, Qt, Console & yes\\
PortAudio 		& GTK, Qt				& yes\\
%\hline
\\
\emph{Mac OS X} \\
%\hline
CoreAudio 		& Qt 			& yes\\
Jack 			& Qt, Console & yes\\
Netjack 			& Qt, Console & yes\\
PortAudio 		& Qt 			& yes\\
%\hline
\\
\emph{Windows} \\
%\hline
Jack 			& Qt, Console & yes\\
PortAudio 		& Qt 			& yes\\
%\hline
\hline
\end{tabular}
\end{center}
\caption{\faust architectures with HTTP support.}
\label{tab:httparch}
\end{table}


\section{A Simple Example}

To illustrate how HTTP support works, let's reuse our previous \code{mix4.dsp} example, a very simplified monophonic audio mixer with four inputs and one output. For each input we have a mute button and a level slider:
\begin{lstlisting}
input(v) = vgroup("input %v", *(1-checkbox("mute")) 
         : *(vslider("level", 0, 0, 1, 0.01)));
process  = hgroup("mixer", par(i, 4, input(i)) :> _);
\end{lstlisting}

We are going to compile this example as a standalone JACK/Qt application with HTTP support using the command:
\begin{lstlisting}
faust2jaqt -httpd mix4.dsp
\end{lstlisting}
The effect of the \code{-httpd} option is to embed a small web server into the application, whose purpose is to serve an HTML page representing its user interface. This page makes use of JavaScript and SVG and closely resembles the native Qt interface.

When we start the application from the command line:
\begin{lstlisting}
./mix4 
\end{lstlisting}
we get various information on the standard output, including:
\begin{lstlisting}
Faust httpd server version 0.72 is running on TCP port 5510
\end{lstlisting}

As we can see, the embedded web server runs by default on TCP port 5510. The entry point is \url{http://localhost:5510}. It can be opened from any recent browser and it produces the page shown in Figure \ref{fig:mix4-http}.


\begin{figure}[h!]
  \centering
  \includegraphics[width=\textwidth]{images/mix4-http.png}
  \caption{User interface of mix4.dsp in a Web browser}   
  \label{fig:mix4-http}
\end{figure}

\section{JSON Description of the User Interface}
The communication between the application and the web browser is based on several underlying URLs. The first one is \url{http://localhost:5510/JSON}, which returns a JSON description of the user interface of the application. This JSON description is used internally by the JavaScript code to build the graphical user interface. Here is (part of) the JSON returned by \code{mix4}:
\begin{lstlisting}
{
  "name": "mix4",
  "address": "YannAir.local",
  "port": "5511",
  "ui": [
    {
      "type": "hgroup",
      "label": "mixer",
      "items": [
        {
          "type": "vgroup",
          "label": "input_0",
          "items": [
            {
              "type": "vslider",
              "label": "level",
              "address": "/mixer/input_0/level",
              "init": "0", "min": "0", "max": "1", 
              "step": "0.01"
            },
            {
              "type": "checkbox",
              "label": "mute",
              "address": "/mixer/input_0/mute",
              "init": "0", "min": "0", "max": "0", 
              "step": "0"
            }
          ]
        },
        
        ...
        
      ]
    }
  ]
}
\end{lstlisting}

\section{Querying the State of the Application}

Each widget has a unique "address" field that can be used to query its value. In our example, the level of input 0 has the address \lstinline'/mixer/input_0/level'. The address can be used to forge a URL to get the value of the widget: \url{http://localhost:5510/mixer/input_0/level}, resulting in:
\begin{lstlisting}
/mixer/input_0/level 0.00000  
\end{lstlisting}

Multiple widgets can be queried at once by using an address higher in the hierarchy. For example, to get the values of the level and the mute state of input 0 we use \url{http://localhost:5510/mixer/input_0}, resulting in:
\begin{lstlisting}
/mixer/input_0/level 0.00000 
/mixer/input_0/mute  0.00000 
\end{lstlisting}

To get all the values at once we simply use \url{http://localhost:5510/mixer}, resulting in:
\begin{lstlisting}
/mixer/input_0/level 0.00000 
/mixer/input_0/mute  0.00000 
/mixer/input_1/level 0.00000 
/mixer/input_1/mute  0.00000 
/mixer/input_2/level 0.00000 
/mixer/input_2/mute  0.00000 
/mixer/input_3/level 0.00000 
/mixer/input_3/mute  0.00000 
\end{lstlisting}

\section{Changing the value of a widget}

\begin{figure}[h!]
  \centering
  \includegraphics[width=\textwidth]{images/mix4-http-mute.png}
  \caption{Muting input 1 by forging the appropriate URL}   
  \label{fig:mix4-http-mute}
\end{figure}

Let's say that we want to mute input 1 of our mixer. We can use the URL \url{http://localhost:5510/mixer/input_1/mute?value=1} obtained by concatenating \url{?value=1} at the end of the widget URL. 

All widgets can be controlled similarly. For example, \url{http://localhost:5510/mixer/input_3/level?value=0.7} will set the input 3 level to 0.7.

\section{Proxy control access to the Web server}

A control application may want to access and control the running DSP using its web server, but without using the delivered HTML page in a browser. Since the complete JSON can be retrieved, control applications can be developed purely in C/C++, then build a \textit{proxy} version of the user interface, and set and get parameters using HTTP requests. 

This mode can be started dynamically using the \textit{-server URL} parameter. Assuming an application with HTTP support is running remotely on the given URL, the control application will fetch its JSON description, use it to dynamically build the user interface, and allow access to the remote parameters.

\section{HTTP cheat sheet}
Here is a summary of the various URLs used to interact with the application's Web server.
\subsection*{Default ports}

\begin{tabular}{ll}
\lstinline'5510' & default TCP port used by the application's Web server\\
\lstinline'5511...' & alternative TCP ports
\end{tabular}

\subsection*{Command-line options}

\begin{tabular}{rl}
\lstinline'-port' $n$ & set the TCP port number used by the application's Web server\\
\lstinline'-server' $URL$ & start a proxy control application accessing the remote application running on the given URL \\
\end{tabular}

\subsection*{URLs}

\begin{tabular}{ll}
\code{http://}\emph{host}\code{:}\emph{port} & the base URL to be used in proxy control access mode \\
\code{http://}\emph{host}\code{:}\emph{port}\code{/JSON} & get a JSON description of the user interface \\
\code{http://}\emph{host}\code{:}\emph{port}\code{/}\emph{address} & get the value of a widget or a group of widgets \\
\code{http://}\emph{host}\code{:}\emph{port}\code{/}\emph{address}\code{?value=}\emph{v} & set the value of a widget to $v$
\end{tabular}

\subsection*{JSON}
\subsubsection*{Top level}
The json describes the name, host and port of the application and a hierarchy of user interface items:
\begin{lstlisting}
{
  "name": <name>,
  "address": <host>,
  "port": <port>,
  "ui": [ <item> ]
}
\end{lstlisting}
An \code{<item>} is either a group (of items) or a widget.

\subsubsection*{Groups}
A group is essentially a list of items with a specific layout: 
\begin{lstlisting}
{
	"type": <type>,
	"label": <label>,
	"items": [ <item>, <item>,...]
}
\end{lstlisting}
The \code{<type>} defines the layout. It can be either \code{"vgroup"}, \code{"hgroup"} or \code{"tgroup"}

\subsubsection*{Widgets}
\begin{lstlisting}
{
	"type": <type>,
	"label": <label>,
	"address": <address>,
	"meta": [ { "key": "value"},... ],
	"init": <num>,
	"min": <num>,
	"max": <num>,
	"step": <num>
},
\end{lstlisting}
Widgets are the basic items of the user interface. They can be of different \code{<type>} : \code{"button"},  \code{"checkbox"}, \code{"nentry"}, \code{"vslider"}, \code{"hslider"}, \code{"vbargraph"} or \code{"hbargraph"}.

